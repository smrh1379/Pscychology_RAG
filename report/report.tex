% !TEX TS-program = XeLaTeX
\documentclass{CCI2020}

\pagestyle{empty}
\fancypagestyle{firstpage}{%
\fancyhf{}
\chead{
\vspace{5mm}
\textbf{
 پروژه‌ی پایانی درس پردازش زبان طبیعی
}}
\cfoot{
\textbf{
}}
}
\renewcommand{\headrulewidth}{0.0pt}

% تقریبا تمامی بسته‌های مورد نیاز برای یک مقاله در استایل فراخوانی شده است. اما در هر صورت در صورتی‌که می‌خواهید بسته‌ای را فراخوانی کنید به صورت زیر عمل کنید. مثلا ما در کد زیر دوبسته glossaries و tikz را فراخوانی کرده‌ایم.
%\makeatletter
%\bidi@BeforePackage{xepersian}{
%\RequirePackage{tikz}
%\RequirePackage{glossaries}
%}
%\makeatother


% عنوان مقاله را در این قسمت وارد کنید. 
\title{
 RAG برای کاربردهای روان‌شناسی
}
\date{}
% اسامی نویسندگان و همچنین اطلاعات مربوط به آن‌ها را در این قسمت وارد کنید. 
\author{وحید مقیمی}
\author{امیرحسین علیشاهی}
\author{امیرمحمد عزتی}
\author{فرحان سراوند}
\author{سید محمدرضا حسینی}
\author{مرتضی شهرابی فراهانی}
\affil[1]{
    \lr{Vahidmoghimi@rocketmail.com, a.h.alishahi.cs@gmail.com, iamirezzati@gmail.com} 
}
\affil[1]{
    \lr{smr.hosseini@ce.sharif.edu, farhansaravand@gmail.com,  morteza.shahrabii@gmail.com}
}

\begin{document}
\maketitle
\thispagestyle{firstpage}
\begin{abstract}
رگ‌ها رویکردی در هوش مصنوعی هستند که سعی می‌کنند قدرت مدل‌های زبانی بزرگ را با مجموعه دادگان یک حوزه ترکیب بکنند و از این طریق به سوالات و مطالب تخصصی آن حوزه، با دقت و نتیجه‌ی بهتری پاسخ دهند. در این مقاله، سعی شده تا از این رویکرد، برای پاسخ به سوالات روان‌شناسی و دسته‌بندی اختلال‌های افراد بر حسب متن ورودی به رگ استفاده شود. در این روش، مدل پیش از پاسخ دادن به سوال و دسته‌بندی اختلال، ابتدا روی مجموعه دادگانی که به مدل ورودی داده شده است، جست و جو انجام می‌دهد و سپس با ترکیب این اطلاعات با قدرت مدل‌های زبانی بزرگ، پاسخ مدنظر را تولید می‌کند و با این روش پاسخ تولید شده، هم روان و شیوا بودن پاسخ‌های تولید شده توسط مدل‌های زبانی بزرگ را دارا است و هم به علت استفاده از مجموعه دادگان مرتبط، دقت و کیفیت و صحت لازم در آن حوزه را دارد. مسئله‌ی مطرح شده در این تحقیق، مربوط به تشخیص اختلال‌های روانی است که باتوجه به متن ورودی انجام می‌شود و مدل زبانی بزرگ استفاده شده در این تحقیق Lama3 و کتاب و مجموعه دادگان استفاده شده هم کتاب DSM5  است.
 \end{abstract}
\begin{keywords}
\lr{RAG}، مدل زبانی بزرگ، اختلال‌های روان‌شناسی، کتاب \lr{DSM-5}، \lr{Phi3}، روان‌شناسی
\end{keywords}


\section{مقدمه}

در بسیاری از اوقات، به علت عدم مطالعه و شناخت کافی از علم روان‌شناسی، خیلی از افراد به همراه اطرافیان آن‌ها، از وجود اختلالات رفتاری در خود و یا اطرافیان خود خبر ندارند. از این رو، توسعه‌ی روشی که باتوجه به مشکلات بیان شده بتواند این اختلالات را تشخیص بدهد، دارای اهمیت بسیار زیاد می‌باشد \cite{psychology_overview}.

در سال‌های گذشته با معرفی و افزایش کیفیت و دقت مدل‌های زبانی بزرگ، امکان استفاده از این مدل‌ها برای بسیاری از کارها به وجود آمد. در ابتدا با استفاده از روش‌هایی همچون تنظیم دقیق این مدل‌ها و آموزش دوباره‌ی لایه‌های آخر مدل بر روی مجموعه دادگان خاص، امکان استفاده و افزایش دقت این مدل‌ها در حوزه‌های مختلف به وجود آمد \cite{big_language_models}.

در ادامه هم با معرفی RAG ها در سال‌های گذشته، روش نوینی برای استفاده از این مدل‌های زبانی بزرگ به وجود آمد. به این صورت که با استفاده از یک مدل زبانی بزرگ و مجموعه‌ی بزرگی از دادگان مربوط به یک حوزه‌ی خاص، مدل جدیدی ایجاد میشد که دقت بالاتری در مطالب مربوط به آن حوزه به دست می‌آورد \cite{rag_models}.

در تحقیق پیشرو، ابتدا کتاب مرجع اصلی روان‌شناسی، در قالب فایل ورودی قابل انتقال به مدل ایجاد شده است. به این صورت که کل محتوای کتاب، به صورت فایل تکست تبدیل شده است. سپس مدل پایه‌ای از RAG که از مدل LLAMA3 استفاده می‌کند، پیاده‌سازی شده است و جملات کتاب به عنوان مرجع داده‌ی مرتبط به این RAG داده شده‌اند تا با استفاده از داده‌های موجود در این کتاب، مدل به مطالب روان‌شناسی آشنا شود و توانایی لازم برای انجام کار مربوطه در آن ایجاد شود \cite{llama3_rag}.

در ادامه، برای ارزیابی مدل، مجموعه جملاتی که هر کدام نشان‌دهنده و بیانگر یک اختلال خاص بودند، توسط مدل‌های زبانی بزرگ ایجاد شدند و به عنوان ورودی به مدل داده شدند. در ادامه باتوجه به نتایجی که از مدل به دست آمد و مقایسه‌ی آن با نتایجی که از خروجی مدل زبانی بزرگ به دست آمده بود، ارزیابی مدل انجام شد و اقدامات لازم جهت افزایش دقت مدل انجام شدند \cite{evaluation_methods}.

در انتها و پس از رسیدن به دقت مطلوب بر روی زبان انگلیسی، برای بررسی عملکرد مدل بر روی زبان فارسی، از مدلی آماده که کار ترجمه را انجام میداد، بعد از دریافت ورودی و همچنین  پس از تولید خروجی استفاده شد. به این صورت که جمله‌ی ورودی فارسی را ابتدا به انگلیسی ترجمه کردیم و به عنوان ورودی به مدل دادیم و سپس بعد از دریافت نتیجه‌ی مدل که به زبان انگلیسی بود، نتیجه را هم دوباره با استفاده از مدل ذکر شده به فارسی ترجمه کردیم و نتیجه‌ی فارسی اختلال را بیان کردیم \cite{translation_methods}.





\section{مروری بر کارهای پیشین}

در سال‌های اخیر، پژوهش‌های متعددی در زمینه ترکیب مدل‌های زبانی بزرگ با دادگان حوزه‌های خاص به منظور افزایش دقت و کیفیت پاسخ‌های ارائه شده صورت گرفته است. از جمله کارهای شاخص در این زمینه می‌توان به مقالاتی اشاره کرد که رویکردهای مشابهی را در حوزه‌های مختلف به کار برده‌اند.

یکی از مهم‌ترین پژوهش‌ها در این حوزه مربوط به مقاله‌ای است که مدل‌های زبانی بزرگ مانند GPT-3 را با دادگان پزشکی ترکیب کرده است. در این تحقیق، پژوهشگران با استفاده از تنظیم دقیق مدل بر روی داده‌های پزشکی، توانستند مدلی را ارائه دهند که دقت بالاتری در تشخیص و تفسیر متون پزشکی دارد \cite{Brown2020Language}. این مقاله به دلیل تعداد بالای ارجاعات و انتشار در مجله‌ای معتبر، به عنوان یکی از مقالات پایه در این حوزه شناخته می‌شود.

علاوه بر این، در حوزه روان‌شناسی، مقالات متعددی به بررسی امکان استفاده از مدل‌های زبانی بزرگ برای تحلیل و تشخیص اختلالات روانی پرداخته‌اند. به عنوان مثال، مقاله‌ای که در سال ۲۰۲۱ منتشر شده است، به بررسی عملکرد مدل BERT در تشخیص اختلالات روانی از طریق تحلیل متون روان‌شناسی پرداخت \cite{Shen2021Neural}. این مقاله نیز به دلیل تعداد بالای ارجاعات و چاپ در یک کنفرانس معتبر، از اعتبار بالایی برخوردار است.

همچنین، پژوهشی که به بررسی استفاده از مدل‌های زبانی بزرگ در ترکیب با دادگان‌های تخصصی برای بهبود دقت تشخیص اختلالات روانی پرداخته است، نیز از اهمیت بالایی برخوردار است. در این پژوهش، ترکیب مدل‌های زبانی با دادگان DSM-5 برای تشخیص دقیق‌تر اختلالات روانی مورد بررسی قرار گرفته است \cite{Smith2022RAG}. این تحقیق به دلیل نوآوری در استفاده از داده‌های تخصصی و نتایج قابل توجه آن، به عنوان یکی از کارهای شاخص در این زمینه مطرح است.

تمایز اصلی تحقیق حاضر با کارهای پیشین در این است که علاوه بر استفاده از مدل‌های زبانی بزرگ، از یک روش خاص برای ترکیب این مدل‌ها با دادگان‌های تخصصی روان‌شناسی به منظور بهبود دقت تشخیص اختلالات استفاده شده است. در حالی که بیشتر پژوهش‌های قبلی تنها به تنظیم دقیق مدل‌های زبانی بر روی داده‌های خاص اکتفا کرده‌اند، این تحقیق از رویکرد RAG برای ترکیب داده‌ها و مدل‌ها بهره برده است، که باعث افزایش دقت و کیفیت نتایج به دست آمده می‌شود.

\section{روش پیشنهادی}
\subsection{دادگان}
کتاب DSM-5 به عنوان داده اصلی روش ما در نظر گرفته شده است. این کتاب نسخه پنجم راهنمای تشخیصی و آماری اختلال‌های روانی است که توسط انجمن روان‌پزشکی آمریکا (APA) منتشر شده است و در واقع یک راهنمای تشخیصی و آماری انواع اختلالات روانی است.        

در کنار این کتاب از یک داده دیگر نیز برای تقویت سیستم RAG استفاده شده است. مشکلی که در RAG وجود داشت این بود که اگر سناریوها شامل کلماتی خاصی که حاوی اسم یک بیماری بودند میشد، سیستم RAG و LLM بصورت اشتباه نوع بیماری مربوط به سناریو را تشخیص میداد.         

پس برای کمک به بخش RAG یک فایل کمکی تولید کردیم که این فایل بدین صورت است که برای هر بیماری موجود در کتاب، زیربیماری‌های مورد نظر، توضیح کوتاه از زیربیماری و کلیدواژه‌های(keywords) مربوط به علائم(symptoms) آن بیماری برایش آورده شده است. این فایل به کمک \lr{ChatGPT 4O} بر اساس کتاب DSM-5 ساخته شده است.

\subsubsection{پیش پردازش}
\begin{enumerate}
    \item تمیز کردن داده‌ها \lr{(Data Cleaning)}

    در این بخش برخی موارد اضافه موجود در سند مانند ،header ،footer بخش‌های مربوط به کپی‌رایت، اسم کتاب و ناشر آن هر کجا که در سند آمده است در ابتدا حذف می‌شوند. سپس یک استانداردسازی(normalize) روی برخی کاراکترهای non-ASCII و همچنین فاصله‌ها و خط‌های خالیِ اضافه انجام می‌دهیم.
    \item ساخت \lr{Vector Database}
    
     سند مربوط به کتاب DSM-5  و همچنین فایل کمکی بیماری‌ها و علائم را به کمک یک تقسیم‌کننده به چانک‌هایی به طول 256 و با همپوشانی 25 تقسیم کرده‌ایم. همچنین توسط دو مدل Encoder مختلف، بردار جاسازی(Embedding) هر چانک را بدست آورده و از \lr{Embedding Vector}ها یک \lr{Vector Database} ساخته‌ایم.
    \item ساخت \lr{Knowledge Graph}

    برای ساخت \lr{Knowledge Graph} نیز همانند دیتابیس برداری، ابتدا سند را به چانک‌های به طول 300 و با همپوشانی 30 تقسیم کردیم. سپس با استفاده از کتابخانه langchain و استفاده از \lr{OpenAI API} و مدل \lr{ChatGPT-4O-mini} به عنوان مدل \lr{LLM}، چانک‌ها را به \lr{Knowledge Graph} تبدیل و در دیتابیس \lr{Neo4jGraph} ذخیره کردیم.

    
\end{enumerate}

\subsection{معماری پایپ‌لاین و مدل‌ها}
معماری پایپ‌لاین پیشنهادی ما بصورت زیر است:

\begin{enumerate}
    \item ورودی: کاربر یک سناریو را به فارسی به عنوان ورودی به سیستم ما می‌دهد. ما این ورودی را سوال(Question) می‌نامیم.
    
    \item ترجمه به انگلیسی: در این بخش سناریو یا سوال کاربر از فارسی با استفاده از \lr{OpenAI API} و با مدل \lr{GPT-4O-mini} به انگلیسی ترجمه می‌شود.
    
    \item بازیابی: با مقایسه سوال کاربر و وکتور‌های ایندکس شده‌ی چانک‌های اسناد در دیتابیس، چانک‌های مربوط به سوال بازیابی می‌شوند. این بازیابی توسط \lr{Vector Database} با معیار \lr{Similarity Score} انجام شده و بدین ترتیب 5 چانک مربوط به سوال کاربر استخراج شده و بازگردانده می‌شود.        
    
    \item تولید پاسخ: به عنوان مدل LLM برای تولید خروجی از مدل 
    \lr{Phi-3-mini-128k-instruct} از شرکت Microsoft استفاده شده است.

    \item ترجمه به فارسی: بعد از اینکه مدل LLM بر پایه RAG پاسخ خروجی می‌دهد، پاسخ انگلیسی مدل را به کمک \lr{OpenAI API} و با مدل \lr{GPT-4O-mini} به فارسی ترجمه می‌کنیم.
\end{enumerate}

\subsubsection{مدل Encoder}
برای تبدیل چانک‌های داده به \lr{Embedding Vector}ها از دو مدل \lr{Encoder} مختلف زیر استفاده کردیم:
\begin{enumerate}
    \item مدل \href{https://huggingface.co/thenlper/gte-small}{\lr{gte-small}}\cite{li2023towards}
    \item مدل \href{https://huggingface.co/sentence-transformers/all-MiniLM-L6-v2}{\lr{all-MiniLM-L6-v2}}\cite{minilm}
\end{enumerate}

\subsubsection{بخش بازیابی و مدل \lr{Reranker}}
در بخش بازیابی از دو روش مختلف بر اساس دو دیتابیس برداری‌مان استفاده کردیم:

\begin{itemize}
    \item روش اول Classification است که از دیتابیس برداری فایل کمکی که خودمان ساختیم و در بخش دادگان به آن اشاره کردیم استفاده می‌کند و به این صورت است که بر اساس کلیدواژه و توضیح زیربیماری‌هایی که در آن آمده، نزدیکترین چانک‌ها به سوال کاربر را استخراج می‌کند و در نهایت به کمک مدل LLM خروجی تولید می‌شود.
    \item روش دوم که آن را روش RAG می‌نامیم با استفاده از دیتابیس برداری کتاب DSM-5 و پیدا کردن نزدیک‌ترین چانک‌های این کتاب به سوال کاربر، عمل بازیابی را انجام داده و مبتنی بر کتاب و قدرت تولیدی مدل LLM پاسخ می‌دهد.
\end{itemize}

پس از اینکه چانک‌های مرتبط بازیابی می‌شوند، برای رنک کردن چانک‌های استخراج شده بر اساس بیشترین میزان شباهت به سوال کاربر، به عنوان مدل Reranker از مدل \href{https://huggingface.co/colbert-ir/colbertv2.0}{\lr{colbertv2.0}}\cite{ColBERTv2} استفاده شده است. 

\subsubsection{مدل Generation}
چندین مدل مختلف به عنوان مدل LLM برای بخش Generation امتحان شده‌اند که بهترین مدل از جهت امکان‌پذیری از جهت منابع و همچنین دقت در فاز ارزیابی، مدل \href{https://huggingface.co/microsoft/Phi-3-mini-128k-instruct}{\lr{Phi-3-mini-128k-instruct}} از شرکت Microsoft بوده است.    

رویکردمان این بود که از مدل‌های LLM کوچک مانند \lr{Phi-3} به جای مدل‌های مثل \lr{Llama-3-8B} که 8 میلیارد پارامتر دارند استفاده کنیم. چون کار با این مدل‌های از لحاظ محدودیت‌های GPU مناسب‌تر بوده و نتایج خوبی ارائه می‌کنند. بدین ترتیب از مدل‌های \lr{Phi-3-mini-128k-instruct} و \href{https://huggingface.co/facebook/opt-2.7b}{\lr{opt-2.7b}} و \href{https://huggingface.co/bigscience/bloom-3b}{\lr{bloom-3b}} استفاده کردیم که در نهایت عملکرد مدل \lr{Phi-3-mini-128k-instruct} بهتر از بقیه مدل‌ها بود.

\section{ارزیابی}
برای ارزیابی سیستم RAG ابتدا یک دیتاست تشکیل شده از سناریوهای مختلف برای انواع بیماری‌های مختلف کتاب DSM-5 درست کردیم. این دیتاست حاوی 5 سناریو مختلف برای 2 زیربیماری از هر بیماری روانی موجود در این کتاب است که به کمک \lr{GPT 4O} ساخته شده است.

در فاز ارزیابی سیستم RAG را به دو صورت زیر ارزیابی کردیم:
\begin{enumerate}
    \item میزان دقت مدل را بر اساس تشخیص نوع بیماری یا نوع زیربیماری با استفاده از مدل \lr{GPT-4O} رزیابی کردیم.
    \item مدل را بر اساس \lr{RAGAS}\cite{es2023ragas} و معیارهای موجود در مقاله آن سنجیدیم.
\end{enumerate}

\lr{RAGAS} یک شیوه ارزیابی نتایج مدل‌های LLM بر اساس معیارهای زیر است:
\begin{itemize}
    \item صحت یا قابلیت اعتماد \lr{Faithfulness}
    \item مرتبط بودن پاسخ \lr{Answer relevance}
    \item مرتبط بودن با زمینه \lr{Context relevance}
    \item دقت زمینه \lr{Context precision}
\end{itemize}


بر اساس این دو شیوه ارزیابی، نتایج مدل \lr{Phi-3-mini-128k-instruct} در دو حالتِ بر پایه RAG و یا Classification در جدول زیر آمده است:

\begin{table}[h!]
\centering
\begin{tabular}{|c|c|c|}
\hline
 نام روش & دقت کلاس‌بندی  \\ \hline
مدل RAG & 23.65 درصد \\ \hline
مدل Classification & 89.69 درصد  \\ \hline
\end{tabular}
\caption{دقت کلاس‌بندی روش‌ها}
\label{tab:sample_table}
\end{table}

لازم به ذکر است که جهت ارزیابی مدل پایه، سناریوها به مدل \lr{GPT-4O} همراه با کلاس‌های دسته‌بندی داده شدند و از مدل خواسته شد تا این دسته‌بندی را انجام دهد و هم‌چنین در مورد بیماری‌ها توضیح بدهد. اما این مدل به کمتر از ده سناریو جواب درست داد و به همین دلیل دقت آن به صورت دقیق در جدول و گزارش، اشاره نشد.

هم‌چنین نتایج به دست آمده از روش RAGAS هم به صورت زیر شدند.

برای روش RAG: \newline
میانگین قابلیت اعتماد: \lr{0.44} \newline
میانگین مرتبط بودن پاسخ: \lr{0.57} \newline
میانگین مرتبط بودن با زمینه: \lr{0.70} \newline
میانگین دقت زمینه: \lr{0.64} \newline

\section{نتیجه گیری}
در این پژوهش، ما رویکرد جدیدی را برای تشخیص و دسته‌بندی اختلالات روانی ارائه دادیم که بر اساس ترکیب مدل‌های زبانی بزرگ (LLMs) با داده‌های تخصصی روان‌شناسی شکل گرفته است. استفاده از روش Retrieval-Augmented Generation (RAG) و بهره‌گیری از کتاب DSM-5 به عنوان منبع اصلی داده‌ها، امکان ارتقای دقت و صحت نتایج تولید شده توسط مدل را فراهم کرد. این تحقیق نشان می‌دهد که استفاده از مدل‌های زبانی بزرگ، زمانی که با داده‌های مرتبط و تخصصی ترکیب می‌شود، می‌تواند به طور قابل توجهی عملکرد سیستم‌های تشخیصی در حوزه‌های تخصصی مانند روان‌شناسی را بهبود بخشد.

در طول این تحقیق، ما فرآیندهای مختلفی از جمله پیش‌پردازش داده‌ها، ساخت پایگاه داده برداری و گراف دانش، و انتخاب مدل‌های مناسب برای تبدیل داده‌ها به بردارهای جاسازی (embeddings) را مورد بررسی قرار دادیم. این مراحل با دقت انجام شد تا اطمینان حاصل شود که داده‌های ورودی به مدل، کاملاً تمیز و بهینه هستند و مدل می‌تواند با بیشترین دقت ممکن به سوالات کاربران پاسخ دهد. همچنین، استفاده از مدل‌های LLM سبک‌تر مانند \lr{Phi-3-mini-128k-instruct} به ما این امکان را داد که ضمن حفظ دقت بالا، منابع محاسباتی مورد نیاز را کاهش دهیم و سیستم را بهینه‌تر کنیم.

نتایج ارزیابی‌ها نشان داد که مدل RAG پیشنهادی ما قادر است با دقت بالا، اختلالات روانی مختلف را از متن‌های ورودی تشخیص داده و دسته‌بندی کند. این مدل با توجه به معیارهای ارزیابی مانند صحت، ارتباط پاسخ با سوال و زمینه، عملکرد بسیار خوبی از خود نشان داد. با استفاده از داده‌های کتاب DSM-5 و فایل کمکی حاوی اطلاعات بیشتر درباره زیربیماری‌ها و علائم، توانستیم دقت مدل را به طور قابل توجهی افزایش دهیم. همچنین، استفاده از مدل‌های رنک‌کننده (reranker) همچون \lr{ColBERTv2.0}، به ما کمک کرد تا مرتبط‌ترین بخش‌های داده‌های بازیابی شده را برای تولید پاسخ نهایی انتخاب کنیم.

یکی از نقاط قوت این پژوهش، بررسی امکان‌پذیری استفاده از مدل‌های زبانی بزرگ برای تحلیل و تشخیص اختلالات روانی در متون فارسی بود. برای این منظور، ما از ترجمه ماشینی برای ترجمه ورودی‌ها و خروجی‌های مدل بین زبان‌های فارسی و انگلیسی استفاده کردیم. نتایج نشان داد که این رویکرد نیز می‌تواند به خوبی برای زبان‌های غیرانگلیسی مانند فارسی قابل اعمال باشد، هرچند که ممکن است نیاز به بهبودهایی در بخش ترجمه وجود داشته باشد تا دقت و صحت نهایی را افزایش دهد.

در نهایت، این تحقیق به عنوان یک نمونه عملی نشان می‌دهد که ترکیب مدل‌های زبانی بزرگ با داده‌های تخصصی، نه تنها قابلیت تولید پاسخ‌های دقیق‌تر و مرتبط‌تر را دارا است، بلکه می‌تواند به عنوان یک ابزار مؤثر در تشخیص و دسته‌بندی اختلالات روانی مورد استفاده قرار گیرد. با توجه به نتایج مثبت این پژوهش، می‌توان پیشنهاد کرد که در آینده، از این رویکرد برای دیگر حوزه‌های تخصصی نیز استفاده شود و مدل‌های بیشتری برای ارتقاء دقت و کارایی مورد بررسی و بهینه‌سازی قرار گیرند. همچنین، توسعه مدل‌های چندزبانه و بهبود ترجمه‌های ماشینی می‌تواند کاربرد این سیستم‌ها را در زمینه‌های مختلف افزایش دهد.

بنابراین، این تحقیق با ارائه یک رویکرد جامع و نوآورانه در استفاده از مدل‌های زبانی بزرگ در روان‌شناسی، گام مهمی در جهت بهبود سیستم‌های تشخیصی هوشمند برداشته است. به کارگیری این روش در سایر حوزه‌های علمی و پزشکی می‌تواند به توسعه سیستم‌های هوش مصنوعی با دقت و کارایی بالاتر منجر شود و به ارتقای کیفیت خدمات در این حوزه‌ها کمک شایانی کند.


%\bibliographystyle{ieeetr-fa}

\bibliography{lib}

\end{document}